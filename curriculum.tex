% Options for packages loaded elsewhere
\PassOptionsToPackage{unicode}{hyperref}
\PassOptionsToPackage{hyphens}{url}
%
\documentclass[
]{article}
\usepackage{lmodern}
\usepackage{amssymb,amsmath}
\usepackage{ifxetex,ifluatex}
\ifnum 0\ifxetex 1\fi\ifluatex 1\fi=0 % if pdftex
  \usepackage[T1]{fontenc}
  \usepackage[utf8]{inputenc}
  \usepackage{textcomp} % provide euro and other symbols
\else % if luatex or xetex
  \usepackage{unicode-math}
  \defaultfontfeatures{Scale=MatchLowercase}
  \defaultfontfeatures[\rmfamily]{Ligatures=TeX,Scale=1}
\fi
% Use upquote if available, for straight quotes in verbatim environments
\IfFileExists{upquote.sty}{\usepackage{upquote}}{}
\IfFileExists{microtype.sty}{% use microtype if available
  \usepackage[]{microtype}
  \UseMicrotypeSet[protrusion]{basicmath} % disable protrusion for tt fonts
}{}
\makeatletter
\@ifundefined{KOMAClassName}{% if non-KOMA class
  \IfFileExists{parskip.sty}{%
    \usepackage{parskip}
  }{% else
    \setlength{\parindent}{0pt}
    \setlength{\parskip}{6pt plus 2pt minus 1pt}}
}{% if KOMA class
  \KOMAoptions{parskip=half}}
\makeatother
\usepackage{xcolor}
\IfFileExists{xurl.sty}{\usepackage{xurl}}{} % add URL line breaks if available
\IfFileExists{bookmark.sty}{\usepackage{bookmark}}{\usepackage{hyperref}}
\hypersetup{
  pdftitle={CV},
  hidelinks,
  pdfcreator={LaTeX via pandoc}}
\urlstyle{same} % disable monospaced font for URLs
\usepackage[margin=1in]{geometry}
\usepackage{graphicx,grffile}
\makeatletter
\def\maxwidth{\ifdim\Gin@nat@width>\linewidth\linewidth\else\Gin@nat@width\fi}
\def\maxheight{\ifdim\Gin@nat@height>\textheight\textheight\else\Gin@nat@height\fi}
\makeatother
% Scale images if necessary, so that they will not overflow the page
% margins by default, and it is still possible to overwrite the defaults
% using explicit options in \includegraphics[width, height, ...]{}
\setkeys{Gin}{width=\maxwidth,height=\maxheight,keepaspectratio}
% Set default figure placement to htbp
\makeatletter
\def\fps@figure{htbp}
\makeatother
\setlength{\emergencystretch}{3em} % prevent overfull lines
\providecommand{\tightlist}{%
  \setlength{\itemsep}{0pt}\setlength{\parskip}{0pt}}
\setcounter{secnumdepth}{-\maxdimen} % remove section numbering

\title{CV}
\author{}
\date{\vspace{-2.5em}}

\begin{document}
\maketitle

First, we define a Lua filter and write it to the file
\texttt{color-text.lua}.

\hypertarget{jose-b.-lanuza}{%
\subsection{Jose B. Lanuza}\label{jose-b.-lanuza}}

Estudiante de doctorado de interacciones planta-polinizador

\href{https://mail.google.com/mail/u/0/\#inbox}{barragansljose@gmail.com}
\includegraphics[width=0.025\textwidth,height=\textheight]{gmaillogo.png}\textbar{}
\href{https://twitter.com/?lang=es}{barragan\_lanuza}
\includegraphics[width=0.025\textwidth,height=\textheight]{twitterlogo.png}\textbar{}
\href{https://github.com/JoseBSL}{JoseBSL}
\includegraphics[width=0.025\textwidth,height=\textheight]{gitlogo.png}

\hypertarget{actualmente}{%
\subsection{ACTUALMENTE}\label{actualmente}}

Cursando un doctorado en la Universidad de New England (Australia)

\hypertarget{educaciuxf3n}{%
\subsection{Educación}\label{educaciuxf3n}}

\texttt{2017-hasta\ la\ fecha} \textbf{Universidad de New England}
Doctorado en Ecología (supervisores: Romina Rader and Ignasi Bartomeus)

\texttt{2015-16} \textbf{Universidad Pablo de Olavide} Máster en
Biodiversidad y Biología de la Conservación

\texttt{2010-15} \textbf{Universidad de Sevilla} Grado en Biología

\hypertarget{presentaciones}{%
\subsection{Presentaciones}\label{presentaciones}}

\texttt{2017\ Sevilla} XIV congreso MEDECOS y XIII encuentro AEET
``Pollinators can change the plant-plant competition regimes'' (January
31-February 4).

\texttt{2020\ Bilbao} XVII ECOFLOR MEETING ``Recipient and donor
characteristics govern the hierarchical structure of heterospecific
pollen competition networks'' (March 4-7).

\hypertarget{publicaciones}{%
\subsection{Publicaciones}\label{publicaciones}}

Lanuza J. B., Bartomeus I., Ashman, T-L., Bible, G., Rader, R. (2021).
Recipient and donor characteristics govern the hierarchical structure of
heterospecific pollen competition networks. Journal of Ecology.

Lanuza, J. B., Bartomeus, I., \& Godoy, O. (2018). Opposing effects of
floral visitors and soil conditions on the determinants of competitive
outcomes maintain species diversity in heterogeneous landscapes. Ecology
Letters, 21(6), 865-874.

\hypertarget{habilidades-tuxe9cnicas}{%
\subsection{Habilidades técnicas}\label{habilidades-tuxe9cnicas}}

\begin{itemize}
\tightlist
\item
  R (Rstudio/Markdown)
\item
  Git
\item
  Species Taxonomy
\item
  Field work experience
\item
  ImageJ
\end{itemize}

\hypertarget{proyectos}{%
\subsection{Proyectos}\label{proyectos}}

\texttt{2017-hasta\ la\ fecha} \emph{Tesis doctoral: Insights of plant
reproduction and species diversity at contrasting scales} Estudio de
competencia por polen entre especies de plantas a través de una
comunidad artificial con floracián simultanea; creación de una bases de
datos con rasgos reproductivos de plantas pertenecientes a redes de
planta-polinizador y análisis de las costes-beneficios de estos rasgos y
la relevancia de los mismos en métricas de red a nivel de especie;
evalucación de patrones planta-polinizador con grupos funcionales de
plantas y subredes (motifs); relaciones entre métricas de condición
espaciales y métricas de campo para mejorar modelos de teledetección.

\texttt{2016-2017} \emph{Tesis de Máster: Factores bióticos y abióticos
pueden modificar los regímenes de competencia planta-planta} Efectos de
la salinidad y los ponilizadores sobre la competencia de especies.

\texttt{2015-2016} \emph{Tesis de Grado: Polinización en plantas
heteroestilas}

\hypertarget{idiomas}{%
\subsection{Idiomas}\label{idiomas}}

\begin{enumerate}
\def\labelenumi{\arabic{enumi}.}
\item
  Español (nativo)
\item
  Inglés (avanzado TOEFL, C1 (Octobre 2016)
\end{enumerate}

\hypertarget{estancias}{%
\subsection{Estancias}\label{estancias}}

\texttt{2017-20} \textbf{Tesis doctoral} (Armidale, Australia)

\texttt{2015-16} \textbf{Técnico de investigación} (\emph{Estación
Bioloógica de Doñana, España}) Trabajos con morfometría de de
polinizadores con cámara Nikon D3300 e ImageJ durante dos meses bajo el
mando del doctor Ignasi Bartomeus.

\texttt{2015-16} \textbf{Técnico de investigación} (\emph{Asturias,
España}) Trabajo de campo durante una semana en el norte de España,
realizando estudios de polinización en el manzano.

\texttt{2013-14} \textbf{Estancia en la Univesidad de Stirling}
(\emph{Stirling, Escocia}) Dos meses en el laboratorio de Mario Vallejo
ayudando a él y sus investigadores a llevar a cabo sus experimentos con
especies del género Mimulus. También trabajo de campo en Shetland
Islands buscando especies de Mimulus.

\texttt{2012-14} \textbf{Alumno interno}(\emph{Universidad de Sevilla,
España}) Dos años en el departamento de botánica con el profesor Juan
Arroyo.

\begin{center}\rule{0.5\linewidth}{0.5pt}\end{center}

Este CV ha sido desarrollado en R Markdown. La versión online en inglés
está disponible en
\href{https://github.com/JoseBSL/CV/blob/master/cv.Rmd}{\textbf{mi
cuenta de github}}
\includegraphics[width=0.025\textwidth,height=\textheight]{gitlogo.png}

\end{document}
